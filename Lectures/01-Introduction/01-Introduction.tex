\documentclass[aspectratio=169]{beamer}

\mode<presentation>
{
  \setbeamertemplate{background canvas}[square]
  \pgfdeclareimage[width=6em,interpolate=true]{dsailogo}{../dsai-logo}
  \pgfdeclareimage[width=6em,interpolate=true]{erasmuslogo}{../erasmus-logo}
  \titlegraphic{\pgfuseimage{dsailogo} \hspace{0.2in} \pgfuseimage{erasmuslogo}}
  %\usetheme{default}
  \usetheme{Madrid}
  \usecolortheme{rose}
  \usefonttheme[onlysmall]{structurebold}
}

\usepackage{pgf,pgfarrows,pgfnodes,pgfautomata,pgfheaps,pgfshade}
\usepackage{amsmath,amssymb}
\usepackage{graphics}
\usepackage{ragged2e}
\usepackage[latin1]{inputenc}
\usepackage{colortbl}
\usepackage[absolute,overlay]{textpos}
\setlength{\TPHorizModule}{30mm}
\setlength{\TPVertModule}{\TPHorizModule}
\textblockorigin{10mm}{10mm}
\usepackage[english]{babel}
\setbeamercovered{dynamic}

\AtBeginSection[]{
  \begin{frame}<beamer>
  \frametitle{Outline}
  \tableofcontents[currentsection]
  \end{frame}
}

\title[Computer Vision]{Computer Vision\\Introduction}
\author{dsai.asia}
\institute[]{Asia Data Science and Artificial Intelligence Master's Program}
\date{}

% My math definitions

\renewcommand{\vec}[1]{\boldsymbol{#1}}
\newcommand{\mat}[1]{\mathtt{#1}}
\renewcommand{\null}[1]{{\cal N}(#1)}
\def\Rset{\mathbb{R}}
\def\Pset{\mathbb{P}}
\def\norm{\mbox{$\cal{N}$}}
\newcommand{\crossmat}[1]{\begin{bmatrix} #1 \end{bmatrix}_{\times}}
\DeclareMathOperator*{\argmax}{argmax}
\DeclareMathOperator*{\argmin}{argmin}
\DeclareMathOperator*{\sign}{sign}
\DeclareMathOperator*{\trace}{trace}

\newcommand{\stereotype}[1]{\guillemotleft{{#1}}\guillemotright}

\newcommand{\myfig}[3]{\centerline{\includegraphics[width={#1}]{{#2}}}
    \centerline{\scriptsize #3}}

\begin{document}

%%%%%%%%%%%%%%%%%%%%%%%%%%%%%%%%%%%%%%%%%%%%%%%%%%%%%%%%%%%%
%%             CONTENTS START HERE

%\setbeamertemplate{navigation symbols}{}

\frame{\titlepage}

%--------------------------------------------------------------------
%\part<presentation>{Part name}
%
%\frame{\partpage}

\begin{frame}
\frametitle{Readings}

Readings for these lecture notes:
\begin{itemize}
\item[-] Szeliski, R. \textit{Computer Vision: Algorithms and Applications},
    2nd ed., \url{https://szeliski.org/Book}, 2021.
\end{itemize}

\medskip

These notes contain material $\copyright$ Hartley and Zisserman
(2004) and Szeliski (2021).

\end{frame}

%--------------------------------------------------------------------
\section{Introduction}
%--------------------------------------------------------------------

\begin{frame}
\frametitle{Introduction}
\framesubtitle{Vision systems}

\myfig{3.5in}{vision-system}{}

The kind of information we want is application specific:

\begin{columns}
\column{2in}
\begin{itemize}
\item 3D models
\item Object categories
\end{itemize}
\column{2in}
\begin{itemize}
\item Object poses
\item Camera poses
\end{itemize}
\end{columns}

\end{frame}

\begin{frame}
\frametitle{Introduction}
\framesubtitle{Applications}

Important applications include:
\begin{itemize}
\item Mobile robot navigation
\item Industrial inspection and control
\item Military intelligence
\item Security
\item Human-computer interaction
\item Image retrieval from digital libraries
\item Medical image analysis
\item 3D model capture for visualization and animation
\end{itemize}

\end{frame}

\begin{frame}
\frametitle{Introduction}
\framesubtitle{Parts of the system}

The ``vision system'' includes:
\begin{itemize}
\item Image \alert{acquisition} hardware
  \begin{itemize}
  \item Analog camera plus digital frame grabber, -or-
  \item Digital camera with a fast serial interface (Firewire, USB,
    etc.)
  \end{itemize}
\item Image processing \alert{support software}
\item Your computer vision \alert{algorithms}
\end{itemize}

\end{frame}

\begin{frame}
\frametitle{Introduction}
\framesubtitle{This summer}

This semester we focus on algorithms for
\begin{itemize}
\item 3D reconstruction
\item Learning (object detection and recognition)
\item Sequential state estimation (e.g.\ tracking, SLAM)
\end{itemize}

\medskip

To understand modern 3D reconstruction techniques we need to
understand how \alert{cameras} transduce the \alert{world} into
\alert{images}.

\medskip

This involves understanding \alert{projective geometry} and
\alert{camera models}.

\medskip

Then we can begin to figure out how to invert what the camera does, or
\alert{reconstruct the 3D scene}.

\end{frame}

\end{document}

